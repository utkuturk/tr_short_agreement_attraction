\documentclass{response}

\usepackage{pdfpages}

\title{Agreement Attraction in Turkish: The case of genitive attractors}
\author{Utku T\"urk \& Pavel Loga\v{c}ev}
\journal{Glossa Psycholinguistics}
%\doi{12345}

\begin{document}

\maketitle

Dear Editors and Reviewers, 

We thank you for your hardwork in preparing the proceedings and giving us the opportunity to submit a revised version of our manuscript entitled \emph{Agreement Attraction in Turkish: The case of genitive attractors}. We highly appreciate the constructive criticisms and valuable comments of the anonymous reviewers on our manuscript. We have revised the manuscript in the submitted version based on the comments of the reviewers. 

Please find below our response to some of the reviewers’ comments. We hope the reviewers find the revised manuscript and our response below to satisfactorily address their comments. 

Please note that reviewers' comments are in bold and italics while our answers are not. Additions to the original manuscript are indicated in blue and deletions with red \emph{only in the response letter}. 

For reference, we add an additional PDF to the end of this response letter in which line numbers are shown and all additions are indicated with red font.

Yours Sincerely,\\
Utku T\"urk, Department of Linguistics, Boğaziçi University\\
Pavel Loga\v{c}ev, Department of Linguistics, Boğaziçi University\\

\section{Reviewer \#1}

\subsection{Clarification of the alternative hypothesis}

\RC My biggest concern is that the predictions of the alternative hypothesis are not clear. I understand the ambiguity, but it’s not clear how the “head noun’s reduced association with subjecthood” as a result of the ambiguity would lead to the contrast observed in Lago et al. The ambiguity would affect all the conditions similarly, so any differences in the ungrammatical conditions for instance (e.g., more “acceptable” responses in the ungrammatical attractor match condition) cannot be reduced to the ambiguity. The authors should step through the details of how their alternative hypothesis predicts the effects observed in Lago et al. 

\AR We thank the reviewer for pointing out the missing part in our argumentation. We agree with the reviewer and decided to add more details on the notion of reduced association with subjecthood. At the end of the Section 2, we dwelled more on the notion of reduced association with subjecthood. We pointed out the syntactic ambiguity and how it would make participants have a different reading of case endings.  

\subsection{Analysis Comparison to Lago et al.}

\RC My only concern with the analysis is more care should be taken to describe how closely the current analysis followed the steps reported in Lago et al. (2019), e.g., did the current study use the same data exclusion techniques, priors, etc.? This is important because if the results were to differ from those reported in Lago et al. (2019), it would not be clear whether the difference is due to the analysis methods or the manipulation. 

\AR 

\subsection{Suggestions}

\RC \ldots the example item set seems well-constructed (although the full list of materials should be provided in the supplemental materials)

\AR 

\RC \ldots the sample size likely provides sufficient power, but a power analysis was not provided.

\AR 

\subsection{Minor Comments}

\RC p. 1: I would suggest a term other than “unrelated” to describe the relation between the attractor and verb. There is a relation, even if indirect. The key point is that the attractor is not the subject of the verb. 

\AR We changed the term ``syntactically unrelated'' to the ``agreement-wise irrelevant.''

\RC p. 1. Typo: agreement controller headed BY key

\AR We included the missing preposition.

\RC Introduction: The authors should note that the studies of interest are on *number* attraction effects. This is important to orient the reader since it’s been recently shown that attraction arises with other forms of agreement as well. 

\AR We included the word ``number'' in the abstract to differentiate the focus of our work from other types of agreement attraction.

\RC p. 3 Typo: due TO case syntcretism

\AR We included the missing preposition.

\RC p. 3: The examples in (4) should have condition labels that match the figure on p. 6

\AR We believe the condition names and the labels on the figures are quite similar. Thus, we did not make any changes.

\RC p. 6 Typo: empty parentheses was identical in both experiments (). 

\AR We corrected the empty parantheses. Now it shows the magnitute that we are discussing.

\section{Reviewer \#2}

\subsection{Clarification of the alternative hypothesis}

\RC I would have liked to see the actual mechanism by which "association with subjecthood" and the case-driven idea is implemented within a cue-based framework like Lewis & Vasishth. At the moment the paper gestures towards the possibility that an implementation is possible, but I suspect that it would be harder than the paper suggests to make the trains run on time. If the authors actually spelled out how they imagine attraction to occur mechanistically with examples, that would be a very nice theoretical contribution. 

\AR 

\subsection{Suggestions}

\RC In the abstract: "Subject head nouns were locally ambiguous between the possessive and the accusative case. We hypothesized that this ambiguity may have inhibited the availability of the head noun as an agreement controller as the accusative is a non-subject case in Turkish." Perhaps a follow up sentence is in order here, explaining that that caused the possessor to be erroneously interpreted as the subject? Or increased the availability of the distractor? 

\AR As suggested by the reviewer, we included a sentence explaining how this ambiguity may serve as a basis for our alternative hypothesis.

\RC I feel that it would be helpful to have a more explicit illustration of the partial match effect that is hypothesized to lead to agreement, wrt feature-matching, case and number. It is not clear to me how "association with subjecthood" is implemented via Case-features. Especially because there are apparently multiple cases that can be associated with subjecthood. Is the proposal that the verb cues for *any* Case that might be associated with subjects (nominative, genitive, possessive...). Or is something else meant? Articulating this would be increase the theoretical contribution of the paper. 

\AR We thank the reviewer for their comment. We now articulated more in the Section 2 on how this association is affected by the syncretism.

\RC At the moment, the description of Lago et al's results, the (terse) explanation for agreement attraction, and the information about the accusative/genitive syncretism are all interwoven in the rather short section 2. I'd recommend separating them out into three distinct parts: Lago et al's results, Lago et al's explanation, Alternative explanation (with discussion of syncretism). 

\AR ASK PAVEL

\RC The explanation would be improved by having a clearer description and concrete examples of the accusative/genitive syncretism. It'd be helpful to have examples with nouns that DON'T exhibit the ambiguity and those that DO exhibit the ambiguity presented side-by-side before we get to the materials. 

\AR TODO

\subsection{Minor Comments}

\RC The phrasing "possessive case marker *coincides in form with* the accusative case marker" is needlessly indirect. Why not just say it's syncretic in certain cases?

\AR We thank the reviewer for the minor comment. We edited the paper as suggested by the reviewer.

\RC The phrasing "we analyzed the data from the present experiment together with Lago et al.’s (2019) data" made me first think that the data were combined. Maybe say the data were compared with the results from Lago et al. 2019? Was Lago et al's data re-analyzed?

\AR We thank the reviewer for their comment. We now have reworded the first sentence in Section 3.4. We indeed re-analyzed Lago et al.'s (2019) data. However, now we emphasize that our intention were to compare our results. 

\RC I'm surprised that Wagers, Lau & Phillips (2009) is not cited in favor of the retrieval position. 

\AR We thank the reviewer for pointing out this deficiency. We wanted to include the Wagers, Lau & Phillips (2009), however, we confused the citation key in bibtex file. It was a mistake on our part. 

\clearpage 



%\includepdf[pages=-]{edits_shown.pdf}
\end{document}
