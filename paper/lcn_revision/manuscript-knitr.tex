% interactapasample.tex
% v1.05 - August 2017

\documentclass[]{interact}\usepackage[]{graphicx}\usepackage[]{color}
% maxwidth is the original width if it is less than linewidth
% otherwise use linewidth (to make sure the graphics do not exceed the margin)
\makeatletter
\def\maxwidth{ %
  \ifdim\Gin@nat@width>\linewidth
    \linewidth
  \else
    \Gin@nat@width
  \fi
}
\makeatother

\definecolor{fgcolor}{rgb}{0.345, 0.345, 0.345}
\newcommand{\hlnum}[1]{\textcolor[rgb]{0.686,0.059,0.569}{#1}}%
\newcommand{\hlstr}[1]{\textcolor[rgb]{0.192,0.494,0.8}{#1}}%
\newcommand{\hlcom}[1]{\textcolor[rgb]{0.678,0.584,0.686}{\textit{#1}}}%
\newcommand{\hlopt}[1]{\textcolor[rgb]{0,0,0}{#1}}%
\newcommand{\hlstd}[1]{\textcolor[rgb]{0.345,0.345,0.345}{#1}}%
\newcommand{\hlkwa}[1]{\textcolor[rgb]{0.161,0.373,0.58}{\textbf{#1}}}%
\newcommand{\hlkwb}[1]{\textcolor[rgb]{0.69,0.353,0.396}{#1}}%
\newcommand{\hlkwc}[1]{\textcolor[rgb]{0.333,0.667,0.333}{#1}}%
\newcommand{\hlkwd}[1]{\textcolor[rgb]{0.737,0.353,0.396}{\textbf{#1}}}%
\let\hlipl\hlkwb

\usepackage{framed}
\makeatletter
\newenvironment{kframe}{%
 \def\at@end@of@kframe{}%
 \ifinner\ifhmode%
  \def\at@end@of@kframe{\end{minipage}}%
  \begin{minipage}{\columnwidth}%
 \fi\fi%
 \def\FrameCommand##1{\hskip\@totalleftmargin \hskip-\fboxsep
 \colorbox{shadecolor}{##1}\hskip-\fboxsep
     % There is no \\@totalrightmargin, so:
     \hskip-\linewidth \hskip-\@totalleftmargin \hskip\columnwidth}%
 \MakeFramed {\advance\hsize-\width
   \@totalleftmargin\z@ \linewidth\hsize
   \@setminipage}}%
 {\par\unskip\endMakeFramed%
 \at@end@of@kframe}
\makeatother

\definecolor{shadecolor}{rgb}{.97, .97, .97}
\definecolor{messagecolor}{rgb}{0, 0, 0}
\definecolor{warningcolor}{rgb}{1, 0, 1}
\definecolor{errorcolor}{rgb}{1, 0, 0}
\newenvironment{knitrout}{}{} % an empty environment to be redefined in TeX

\usepackage{alltt}

\usepackage{epstopdf}% To incorporate .eps illustrations using PDFLaTeX, etc.
\usepackage[caption=false]{subfig}% Support for small, `sub' figures and tables
%\usepackage[nolists,tablesfirst]{endfloat}% To `separate' figures and tables from text if required
\usepackage[doublespacing]{setspace}% To produce a `double spaced' document if required
\setlength\parindent{24pt}% To increase paragraph indentation when line spacing is doubled


\usepackage{tikz,xcolor,hyperref}

% Make Orcid icon
\definecolor{lime}{HTML}{A6CE39}
\DeclareRobustCommand{\orcidicon}{%
	\begin{tikzpicture}
	\draw[lime, fill=lime] (0,0) 
	circle [radius=0.16] 
	node[white] {{\fontfamily{qag}\selectfont \tiny ID}};
	\draw[white, fill=white] (-0.0625,0.095) 
	circle [radius=0.007];
	\end{tikzpicture}
	\hspace{-2mm}
}

\foreach \x in {A, ..., Z}{%
	\expandafter\xdef\csname orcid\x\endcsname{\noexpand\href{https://orcid.org/\csname orcidauthor\x\endcsname}{\noexpand\orcidicon}}
}

% Define the ORCID iD command for each author separately. Here done for two authors.
\newcommand{\orcidauthorA}{0000-0001-8011-1541}
\newcommand{\orcidauthorB}{0000-0002-2837-2258}


%\usepackage[longnamesfirst,sort]{natbib}% Citation support using natbib.sty
%\bibliographystyle{unsrtnat}
%\bibpunct[, ]{(}{)}{;}{a}{,}{,}% Citation support using natbib.sty
%\renewcommand\bibfont{\fontsize{10}{12}\selectfont}% To set the list of references in 10 point font using natbib.sty

\usepackage[natbibapa,nodoi]{apacite}% Citation support using apacite.sty. Commands using natbib.sty MUST be deactivated first!
\setlength\bibhang{12pt}% To set the indentation in the list of references using apacite.sty. Commands using natbib.sty MUST be deactivated first!
%\let\cite\shortcite

\renewcommand\bibliographytypesize{\fontsize{10}{12}\selectfont}% To set the list of references in 10 point font using apacite.sty. Commands using natbib.sty MUST be deactivated first!

\theoremstyle{plain}% Theorem-like structures provided by amsthm.sty
\newtheorem{theorem}{Theorem}[section]
\newtheorem{lemma}[theorem]{Lemma}
\newtheorem{corollary}[theorem]{Corollary}
\newtheorem{proposition}[theorem]{Proposition}

\theoremstyle{definition}
\newtheorem{definition}[theorem]{Definition}
\newtheorem{example}[theorem]{Example}

\theoremstyle{remark}
\newtheorem{remark}{Remark}
\newtheorem{notation}{Notation}

\let\B\relax
\let\T\relax
\usepackage{xyling} 
\usepackage[linguistics]{forest}
\usepackage{linguex}
\renewcommand{\firstrefdash}{}%
\AtBeginDocument{\settowidth{\Exlabelwidth}{(110)}}
\RequirePackage{cgloss} %for adding the language name and source of the example on the first line of glossed examples (requires \gll before the foreign language example and \glt before the translation)

\newcommand{\rev}[1]{{\color{red}#1}}
\IfFileExists{upquote.sty}{\usepackage{upquote}}{}
\begin{document}

\articletype{Short Article}% Specify the article type or omit as appropriate

\title{Agreement Attraction in Turkish: The Case of Genitive Attractors}

\author{
\name{Utku T\"{u}rk\textsuperscript{a}\orcidA{}\thanks{CONTACT Utku T\"{u}rk. Email: utkuturk@pm.me} and Pavel Loga\v{c}ev\textsuperscript{a}\orcidB{}}
\affil{\textsuperscript{a}Bo\u{g}azi\c{c}i University, \.{I}stanbul, Turkey}
}

\maketitle

\begin{abstract}
Speakers have been shown to find sentences with erroneous agreement acceptable under certain conditions. This so-called agreement attraction effect has also been found in genitive-possessive structures such as \textit{``the teacher's brother''} in Turkish (Lago et al., 2019), which is in contrast to its absence in similar constructions in English (Nicol et al., 2016). It has been hypothesized that this discrepancy is a result of the association between genitive case marking and subjecthood in Turkish. We test an alternative explanation according to which Lago et al.'s findings are due to a potential confound in their experiment, as all subject head nouns were locally ambiguous between possessive and accusative case. The results of our speeded acceptability judgment experiment suggest that the presence of case ambiguity does not affect agreement attraction, and thus suggests that genitive NPs may function as attractors in Turkish due to the association between genitive case and subjecthood.



\end{abstract}

\begin{keywords}
Agreement attraction; syncretism; Turkish; \rev{sentence processing; number}
\end{keywords}


\section{Introduction}

Speakers often fail to accurately process grammatical dependencies between different parts of a sentence \citep[e.g.,][]{GibsonThomas:1999,PhillipsEtAl:2011}. For example, in \ref{ex:pearlmutter}, the auxiliary verb \textit{were} erroneously agrees with the agreement-wise irrelevant attractor noun phrase headed by \textit{cabinets} instead of the agreement controller headed by \textit{key}. A number of previous studies in comprehension \citep{NicolEtAl:1997, PearlmutterGarnseyBock:1999, WagersEtAl:2009} showed that participants found sentences like \ref{ex:pearlmutter} acceptable more often and read them faster compared to their counterparts with a singular attractor. This phenomenon, known as \textit{agreement attraction} \citep{BockMiller:1991} has been attested in a number of languages, such as in Arabic \citep{TuckerEtAl:2015}, Armenian \citep{AvetisyanEtAl:2020}, German \citep{LagoFelser:2018}, Hindi \citep{BhatiaDillon:2020}, Serbian \citep{RisticEtAl:2016}, Slovak \citep{BadeckerKuminiak:2007}, Spanish \citep{LagoEtAl:2015}, and recently in Turkish \citep{LagoEtAl:2019}.


\ex. \label{ex:pearlmutter} * The \underline{key} to the cabinets \underline{were} rusty from many years of disuse. 


\citet{LagoEtAl:2019} demonstrated that genitive possessors (such as \textit{painters} in \textit{the painters' rival}) effect agreement attraction effects in Turkish. However, this finding appears to be at odds with \citeauthor{NicolEtAl:2016}'s (\citeyear{NicolEtAl:2016}) results, who failed to find a similar effect in English. \citet{LagoEtAl:2019} hypothesize that Turkish possessor noun phrases, unlike their English counterparts, may function as agreement attractors because Turkish genitive NPs may function as subjects of non-finite clauses in Turkish \citep{GokselKerslake:2005,Kornfilt:2011}. As a result, Turkish genitive NPs may match the subjecthood feature used in the cue-based retrieval of a verb's subject \citep{LewisVasishth:2005, WagersEtAl:2009, ArnettWagers:2017}.

In this paper, we test an alternative explanation of \citeauthor{LagoEtAl:2019}'s (\citeyear{LagoEtAl:2019}) findings which is related to an instance of local ambiguity in \rev{their} experimental sentences, as a result of which all subject head nouns were ambiguous between possessive and accusative case.


\section{Agreement Attraction in Turkish}

Turkish is an SOV word order language with overt case marking \citep{GokselKerslake:2005,Kornfilt:2011,Kornfilt:2013}. 
For example, the possessor in the Turkish construction in \ref{ex:possgenex}, which is similar to the Saxon Genitive, is marked with genitive case, while the head noun is marked with possessive case. 
Importantly, the possessive case marker coincides in form with the accusative case marker when the noun stem ends in a consonant:
This is because in Turkish, case markers have different forms depending on whether the last sound of the stem is a vowel. In order to break up vowel-vowel clusters, Turkish uses epenthetic consonants such as \textit{s}, \textit{y}, or \textit{n}.  For example, the genitive case can surface as \textit{-nin} or as \textit{-in}, depending on the last consonant of the stem. The forms of the possessive marker and the accusative case are identical except for the epenthetic consonant. This means that they surface as \textit{-i} in consonant-ending words, but as \textit{-si} and \textit{-yi} respectively in vowel-ending words. 

\ex. \label{ex:possgenex}
\gll $[$$[$teknisyen-in$]$ e\u{g}itmen-i$]$ \\
    technician-\textsc{gen} instructor-\textsc{poss} \\
\glt ``the instructor of the technician''


\citet{LagoEtAl:2019} demonstrated an agreement attraction effect in Turkish genitive-possessive constructions in a speeded acceptability judgment study with sentences like \ref{ex:lago}, in which the number of the attractor and the verb was manipulated. The resulting $2$x$2$ design, indicated by brackets and slashes in \ref{ex:lago}, consisted of two grammatical conditions, in which the verb agreed \rev{in number} with the singular subject head noun, and two ungrammatical conditions, in which the verb carried plural agreement, and thus did not agree with the subject. In ungrammatical sentences, they found a higher percentage of \textit{acceptable} responses when the genitive attractor was plural than when it was singular, indicating agreement attraction. No such effect was found in grammatical sentences.


\ex. \label{ex:lago}
\gll Teknisyen-\{ler/\O\}-in e\u{g}itmen-i ola\u{g}an{\"u}st{\"u} h{\i}zl{\i} ko\c{s}-tu-\{lar/\O\}.\\
technician-\{\textsc{pl}/\textsc{sg}\}-\textsc{gen} instructor-\textsc{poss} extraordinarily fast run-\textsc{pst}-\{\textsc{pl}/\textsc{sg}\}\\
\glt ``The technician's/technicians' instructor \rev{ran\{\textsc{pl}/\textsc{sg}\}} extraordinarily fast.''


\citet{LagoEtAl:2019} hypothesized that these effects originated from how case and number information \rev{are} encoded and retrieved: According to \citeauthor{LewisVasishth:2005}'s (\citeyear{LewisVasishth:2005}) cue-based retrieval model, phrases are encoded in a content-addressable memory as bundles of features called \textit{chunks} which include information like number, gender, case, and syntactic function \citep[e.g.,][]{SmithVasishth:2020}. Under \citeauthor{LagoEtAl:2019}'s (\citeyear{LagoEtAl:2019}) proposal, participants predict the number of the verb based on the noun phrases they process while reading the subject. In grammatical sentences with singular verb agreement, the number prediction and the verb number match, which causes no processing difficulty. In contrast, when participants fail to find the predicted number morphology on the verb, a memory-retrieval process is initiated. This process activates the search for a chunk matching two cues: the subjecthood feature ([+SUBJECT]) and the plural feature ([+PL]). While neither of the available noun phrases matches this specification in ungrammatical agreement attraction sentences, each of the NPs headed by \textit{technician} and \textit{instructor} matches one of these cues. While this partial match mostly results in participants finding the sentence ungrammatical, they may retrieve the attractor \textit{technicians} on some trials. \citet{LagoEtAl:2019} argue that this erroneous retrieval may be facilitated by the genitive case marking on the attractor, because genitive NPs can function as subjects of embedded clauses in Turkish. Due to the ubiquity of genitive subjects, attractors marked with genitive case are a priori likely to be agreement controllers.

However, a potential problem with the stimuli in the \citet{LagoEtAl:2019} study is that the stems of all head nouns such as \textit{e\u{g}itmen-i} (\textit{`instructor'}) in \ref{ex:lago} were consonant-ending, making their possessive forms ambiguous between possessive and accusative case due to case syncretism between them in vowel-ending stems \citep[pp. 66--67]{GokselKerslake:2005}. As a result, the head noun (\textit{'instructor'}) in \ref{ex:ambiguous} can be disambiguated towards possessive case as in \ref{ex:ambiguous_possessive}, or towards accusative case as in \ref{ex:ambiguous_accusative}, where the genitive-marked noun functions as an embedded subject, and the matrix subject is omitted due to pro-drop. 
\rev{We also calculated the relative probability of having an accusative marked noun compared to an possessive marked noun following genitive marking. Data from annotated treebanks from Universal Dependencies v2.9 \citep{TurkEtAl2021,KuzgunEtAl2020,TurkEtAl2019b,SulubacakEtAl2016,Coltekin2015} showed that the relative probability of encountering accusative marking after the genitive-marked noun is 0.21.} 

\ex.
\label{ex:ambiguous}
  \a. \textsc{Possessive Interpretation} \label{ex:ambiguous_possessive} \\
  \gll Teknisyen-in e\u{g}itmen-i ko\c{s}-tu.\\
       technician-\textsc{gen} instructor-\textsc{poss} run-\textsc{pst}\\
  \glt ``The technician's instructor ran.''
  \b. \textsc{Accusative Interpretation} \label{ex:ambiguous_accusative}\\
  \gll Teknisyen-in e\u{g}itmen-i kov-du\u{g}-un-u g\"{o}r-d\"{u}-m.\\
  technician-\textsc{gen} instructor-\textsc{poss} fire-\textsc{nmlz}-\textsc{poss}-\textsc{acc} see-\textsc{pst}-\textsc{1sg}\\
  \glt ``I saw the technician firing the \rev{instructor}.''

Because accusative NPs cannot function as subjects in Turkish, it is possible that Lago et al.'s finding of agreement attraction effects in sentences like \ref{ex:lago} are not due to the genitive attractors' association with subjecthood, but rather due to the head nouns' \textit{reduced association with subjecthood} due to its ambiguity.

Because sentences like \ref{ex:lago} are locally ambiguous, participants may initially adopt an incorrect analysis of the genitive-possessive structure on some trials, and encode the genitive-marked first noun as the subject of an embedded verb and the ambiguous second noun as an accusative object. 
Under the assumption that remnants of an incorrect analysis affect the parsing process even after after a successful reanalysis \citep{Staub:2007}, the initial association between the genitive noun phrase and subjecthood may lead an agreement attraction effect. 
In contrast, this account predicts that the agreement attraction effect should be either significantly reduced or entirely absent when the head noun is unambiguous, as the function of the genitive noun phrase is disambiguated early, thus preventing potential digging-in effects \citep{Tabor:2004}. We tested this hypothesis in a speeded-acceptability experiment with sentences similar to \citeauthor{LagoEtAl:2019}'s (\citeyear{LagoEtAl:2019}), but with unambiguously marked vowel-ending head nouns. 

\section{The Present Study}

The present study tested the predictions of the case syncretism account as an alternative explanation of the previously found agreement attraction effect in Turkish. To avoid the ambiguity present in \citet{LagoEtAl:2019}, we used vowel-ending head nouns nouns such as \textit{terzi} (\textit{`tailor'}), for which the possessive marker surfaces as \textit{-si} and is distinct from the accusative (\textit{-yi}). We hypothesized that if the morpho-phonological ambiguity was a key factor in agreement attraction in Turkish, unambiguous sentences like ours in \ref{ex:our_items} should not elicit attraction effects.

\ex. \label{ex:our_items}
  \a. * \textsc{Plural Attractor, Ungrammatical (Plural Verb)} \label{ex:expitem-plpl}\\
  \gll [Milyoner-ler-in \textbf{terzi-si}] tamamen gereksizce \textbf{kov-ul-du-lar}.\\
  millionaire-\textsc{pl}-\textsc{gen} tailor-\textsc{poss} completely without.reason fire-\textsc{pass}-\textsc{pst}-\textsc{pl}\\
  \glt ``The millionaires' tailor were fired for no reason at all.''
  \b. \textsc{Plural Attractor, Grammatical (Singular Verb)} \label{ex:expitem-plsg} \\
  \gll [Milyoner-ler-in \textbf{terzi-si}] tamamen gereksizce \textbf{kov-ul-du}.\\
  millionaire-\textsc{pl}-\textsc{gen} tailor-\textsc{poss} completely without.reason fire-\textsc{pass}-\textsc{pst}\\
  \glt ``The millionaires' tailor was fired for no reason at all.''
  \b. * \textsc{Singular Attractor, Ungrammatical (Plural Verb)} \label{ex:expitem-sgpl}\\
  \gll [Milyoner-in \textbf{terzi-si}] tamamen gereksizce \textbf{kov-ul-du-lar}.\\
  millionaire-\textsc{gen} tailor-\textsc{poss} completely without.reason fire-\textsc{pass}-\textsc{pst}-\textsc{pl}\\
  \glt ``The millionaire's tailor were fired for no reason at all.''
  \b. \textsc{Singular Attractor Grammatical (Singular Verb)} \label{ex:expitem-sgsg}\\
  \gll [Milyoner-in \textbf{terzi-si}] tamamen gereksizce \textbf{kov-ul-du}.\\
  millionaire-\textsc{gen} tailor-\textsc{poss} completely without.reason fire-\textsc{pass}-\textsc{pst}\\
  \glt ``The millionaire's tailor was fired for no reason at all.''
  

\subsection{Participants} 


We recruited 118 undergraduate students to participate in the experiment in exchange for course credit. All participants were native Turkish speakers, with an average age of 20 (range: 18 -- 32). The experiment was carried out following the principles of the Declaration of Helsinki and the regulations concerning research ethics at Bo\u{g}azi\c{c}i University. All participants provided informed consent before their participation and their identity \rev{were} completely anonimised.

\subsection{Materials}

We used 40 sets of sentences like \ref{ex:our_items}, in which we manipulated (i) the number of the attractor noun and (ii) the number agreement on the verb. Plural number and plural agreement were both marked with the suffix \textit{-ler/-lar}, while the singular number and singular agreement were marked by its absence. We used the experimental items from \citet{LagoEtAl:2019} as a starting point for all items. We substituted ambiguous nouns for unambiguous alternatives, and in some cases, modified other parts of the sentence for plausibility reasons.

All sentences started with a complex subject NP like \textit{milyonerlerin terzisi} ``\textit{the millionaires' tailor},'' in which the genitive possessor functioned as the attractor, and the head noun carried an unambiguous possessive case marker. Because the plural marking on nominals is not optional and the head noun was singular, absent of \textit{-lar}, in all conditions, sentences with plural verb agreement were ungrammatical. Moreover, the semantic relationship between the possessor and the head noun was kept as it is in \citeauthor{LagoEtAl:2019}'s (\citeyear{LagoEtAl:2019}) original study and genitive-possessive structures can be paraphrased using \textit{'s} or \textit{of} in English. The distribution of the verb types matched that of the original study, with twenty unergatives, eighteen unaccusatives, and two optionally transitive verbs. Pre-verbal adverbials also consisted of 2-3 words (15 characters on average).

One example set of experimental items is in \ref{ex:our_items}. The subject phrase is marked with square brackets, and the dependency between the subject head and the matrix verb is signaled using bold-face.


We hypothesized that the experimental sentences in \ref{ex:our_items} might elicit a simple response strategy based on verb number because all ungrammatical sentences end with a plural-agreement-bearing verb. In contrast, all grammatical sentences end with a verb that lacks plural agreement, thus singular. As a result, some participants may resort to classifying sentence acceptability based on their last word after repeated exposure to sentences like \ref{ex:our_items}. In order to preclude such a response strategy, we designed 40 filler sentences that would render it ineffective. We included 20 grammatical sentences like \ref{ex:gram_filler} with plural and 20 ungrammatical sentences like \ref{ex:ung_filler} with a singular verb. Filler items resembled experimental sentences in that they started with a complex genitive-possessive noun phrase. In contrast to the experimental items, however, the complex NPs were the subject of an adverbial clause instead of the main sentence. In grammatical fillers like \ref{ex:gram_filler}, we have used pro-dropped subjects, which enabled us to use plural verbs without having ungrammatical sentences. 


\ex. \label{ex:fillers}
  \a. \textsc{Grammatical Filler (Plural Verb)} \label{ex:gram_filler}\\
  \gll [Sosyolog-un \textbf{\"{o}\u{g}renci-si}] konu\c{s}-unca tutars{\i}zl{\i}k a\c{c}{\i}\u{g}-a \textbf{\c{c}{\i}kar-d{\i}-lar}.\\ 
  sociolog-\textsc{gen}  student-\textsc{poss} speak-\textsc{when} inconsistency  open-\textsc{dat} deduct-\textsc{pst}-\textsc{pl}\\
  \glt ``When the student of the sociologist spoke, they revealed an inconsistency.''
  \b. * \textsc{Ungrammatical Filler (Singular Verb)} \label{ex:ung_filler}\\
  \gll [Dans\"{o}z-\"{u}n \textbf{koca-s{\i}}] var-{\i}nca kap{\i} sakince \textbf{a\c{c}-t{\i}}. \\
  dancer-\textsc{gen}  husband-\textsc{poss} arrive-\textsc{when} door slowly  open-\textsc{pst}\\
  \glt Intended:``When the husband of the dancer came, the door opened slowly.''


\subsection{Procedure}

The experiment was run online, using the web-based platform Ibex Farm \citep{Drummond2013}. Each experimental session took approximately 25 minutes to complete. Participants provided demographic information and gave informed consent to participate in the experiment. They then proceeded to read the instructions and were given nine practice trials before the experiment began.

Each trial began with a blank screen for 600 ms, followed by a word-by-word RSVP presentation of the sentence in the center of the screen, followed by a prompt to indicate their acceptability judgment. Sentences were presented word-by-word in the center of the screen in 30 pt font size, at a rate of 400 ms per word. Participants saw a blank screen for 100 ms between each word, and to see the next item, they needed to press the \texttt{space} key. Participants were asked to press the key \texttt{P} to indicate that a sentence is acceptable and \texttt{Q} to indicate that the sentence is unacceptable. They were instructed to provide judgments as quickly as possible. During the experiment, a warning message in red font appeared if they did not respond within 5,000 ms.

Participants saw 40 experimental and 40 filler sentences. Experimental sentences were distributed among four different lists according to a Latin-square design. Every participant saw one version of the experiment with a specific list and one item per condition.

\subsection{Analysis}

In order to test whether the morphological ambiguity present in the \citet{LagoEtAl:2019} sentences affected the presence or magnitude of the agreement attraction effect, we analyzed the data from the present experiment \rev{and} compared our results to \citeauthor{LagoEtAl:2019}'s (\citeyear{LagoEtAl:2019}) results, by including \citeauthor{LagoEtAl:2019}'s (\citeyear{LagoEtAl:2019}) data in our Bayesian GLM and using the experiment as an additional factor in the analysis.
  
Prior to the analysis, we removed the data for all participants who failed to show sufficient sensitivity to the effect of grammaticality in singular attractor conditions, i.e., when no agreement attraction was expected. Specifically, we removed all participants for whom the difference in the percentage of \textit{yes} responses between \rev{conditions with singular attractors \ref{ex:expitem-sgsg} and \ref{ex:expitem-sgpl}} fell below the threshold of 0.1 percentage points. We also excluded trials in which the participants missed the response deadline. As a result, we excluded \rev{3.39\%} of the trials from our experiment and 2.27\% of the \citeauthor{LagoEtAl:2019}'s (\citeyear{LagoEtAl:2019}) trials. 

We analyzed responses using two Bayesian GLMs assuming a Bernoulli-distributed response with a probit link function.  We used the R packages brms \citep{brms} and rstan \citep{rstan} to fit Bayesian hierarchical models \citep[e.g.,][]{GelmanHill:2007, NicenboimVasishth:2016}. 
We analyzed only experimental sentences and used (i) grammaticality of the sentence, (ii) attractor number, and (iii) presence of morphological ambiguity (i.e., experiment), as well as all their interactions as predictors. We used by-participant and by-item intercepts and slopes for all predictors and \rev{their interaction}. 
All factors were sum-coded. \rev{A Student's \emph{t}-distribution ($\nu$ = 3, $\mu$ = 0, $\sigma$ = 2.5), a normal distribution ($\mu$ = 0, $\sigma$ = 1), a Cauchy distribution (($\mu$ = 0, $\sigma$ = 1)) and a LKJ distribution ($\eta$ = 2) were used for our intercept, slopes of predictors, standard deviations of random effects, correlation coefficients in interaction models respectively as priors.}

Because the magnitude of the agreement attraction effect can be operationalised either as the interaction between grammaticality and the presence of a plural attractor, or as the effect of a plural attractor in ungrammatical sentences, we used a second model with predictors (ii) and (iii) and their interaction to analyze responses to ungrammatical conditions only.


In the results section, we provide summaries of the coefficient posterior distributions. For our model, we ran 4 chains with 1000 warm-up iterations and 1000 sampling iterations. The data for our study, along with our analysis scripts can be found at \url{https://osf.io/hegmd/}.

\subsection{Results}

Figure \ref{fig:AverageResponses} shows the average proportions of ``acceptable'' responses by experimental condition in our experiment with unambiguous possessive marking, side by side with Lago et al.'s findings. It shows that ungrammatical sentences with plural attractors are rated as acceptable more often (M = 0.24, SE = 0.01) than their counterparts with singular attractors (M = 0.14, SE=0.01). The magnitude of the effect (0.10) was in line with the findings reported in \citet{LagoEtAl:2019}, where the difference was also 0.11. Accuracy rates for grammatical conditions were nearly equal (M = 0.92 and 0.91, SE = 0.01 and 0.01, for singular and plural attractors respectively).


\begin{knitrout}
\definecolor{shadecolor}{rgb}{0.969, 0.969, 0.969}\color{fgcolor}\begin{figure}[hbt!]

{\centering \includegraphics[width=\linewidth]{figure/AverageResponses-1} 

}

\caption{The average percentage of acceptable responses according to the experimental conditions in our study and \citet{LagoEtAl:2019}. Error bars signal standard errors calculated following \citet{Morey:2008,Cousineau:2005}.}\label{fig:AverageResponses}
\end{figure}

\end{knitrout}


Figure \ref{fig:ResponseModel} shows estimates and 95\% credible intervals of a Bayesian GLM with a probit link function. The main effect of grammaticality ($\hat{\beta}=3.01;$ $CI=[2.76; 3.27];$ $P(\beta<0)< .001$) indicates that, on average, participants were quite good at distinguishing between grammatical and ungrammatical sentences. Meanwhile, the negative interaction between grammaticality and attractor number ($\hat{\beta}=-0.71;$ $CI=[-1.00; -0.44];$ $P(\beta<0)> .999$) indicated a larger difference (positive) effect of plural attractors in ungrammatical conditions, and thus a number agreement attraction effect. There was weak evidence for a negative three-way interaction between the presence of ambiguity, ungrammaticality, and attractor number ($\hat{\beta}=-0.29;$ $CI=[-0.81; 0.20];$ $P(\beta<0)=    .88$), which was largely driven by differences in the effect of attractor number in \textit{grammatical conditions}, as the magnitude of the effect in the ungrammatical conditions was identical in both experiments ().  
This is consistent with the estimates of the model based on ungrammatical sentences in Figure \ref{fig:ResponseModelUngram}, which show no indication of an interaction between ambiguity and the presence of a plural attractor (($\hat{\beta}=0.06;$ $CI=[-0.25; 0.37];$ $P(\beta<0)=    .36$). It also showed a main effect of plural attractor ($\hat{\beta}=0.50;$ $CI=[0.32; 0.67];$ $P(\beta<0)< .001$). Taken together, the coefficients indicated a substantial agreement attraction effect regardless of the presence of local ambiguity.  


\begin{knitrout}
\definecolor{shadecolor}{rgb}{0.969, 0.969, 0.969}\color{fgcolor}\begin{figure}[hbt!]

{\centering \includegraphics[width=\linewidth]{figure/ResponseModel-1} 

}

\caption{Estimates and 95\% credible intervals for the probit regression coefficients for the model of responses in our experiment and \citet{LagoEtAl:2019}.}\label{fig:ResponseModel}
\end{figure}

\end{knitrout}

\begin{knitrout}
\definecolor{shadecolor}{rgb}{0.969, 0.969, 0.969}\color{fgcolor}\begin{figure}[hbt!]

{\centering \includegraphics[width=\linewidth]{figure/ResponseModelUngram-1} 

}

\caption{Estimates and 95\% credible intervals for the probit regression coefficients for the model of responses to \protect{\textbf{ungrammatical sentences}} in our experiment and \citet{LagoEtAl:2019}.}\label{fig:ResponseModelUngram}
\end{figure}

\end{knitrout}


\section{Discussion \& Conclusion}

We re-examined the findings of \citet{LagoEtAl:2019} and investigated the contributution of a possible confound to their finding of an agreement attraction effect in genitive-possessive constructions in Turkish. Our main question was whether \citeauthor{LagoEtAl:2019}'s (\citeyear{LagoEtAl:2019}) findings can be explained by an alternative hypothesis: 
Because in their experimental sentences, all head nouns were locally ambiguous between the possessive and the accusative case (a non-subject case in Turkish), we hypothesized that this may have weakened the strength of association between the subject head noun and the subjecthood feature.
If Turkish agreement attraction effects in genitive-possessive structures resulted from this ambiguity, we expected the absence of agreement attraction effects when the case of the head noun was unambiguous.  

Our experimental findings were comparable to \citet{LagoEtAl:2019}: We observed that the presence of a plural attractor increased the rate of erroneous ``acceptable'' responses  
% increased the overall error rate
in ungrammatical sentences.
Importantly, we did not find an effect of case ambiguity: While our model based on all experimental conditions indicated a weak three-way interaction between ambiguity, grammaticality, and attractor number, a model based on ungrammatical sentences only demonstrated that this is due to a difference in acceptability rates in grammatical conditions. This model showed no evidence of an interaction between attractor number and ambiguity, indicating no evidence of a modulation of the effect size of the agreement attraction effect by the presence of a local case ambiguity. 
Although the $95\%$ credible interval for the interaction term was relatively wide, our findings indicate the presence of a substantial agreement attraction effect regardless of the presence of local ambiguity.
Thus, we successfully replicated the findings of \citet{LagoEtAl:2019} with disambiguated head nouns. 

Taken together, our results suggest (i) that agreement attraction effects in Turkish are not due to a reduced association between case-ambiguous nouns and the subjecthood features, and (ii) that local ambiguities, such as case syncretism, do not appear to play a role in agreement attraction. Participants do not appear to rely on form-related cues in their decision-making processes and instead use of abstract linguistic features as retrieval cues.

\section*{Abbreviations} %exclude


\textsc{dat} = dative, \textsc{gen} = genitive, \textsc{nmlz} = nominaliser, \textsc{pass} = passive, \textsc{pl} = plural, \textsc{poss} = possessive, \textsc{pst} = past, \textsc{sg} = singular, \textsc{when} = when. %exclude


\section*{Authors' Contributions} %exclude


Utku T\"urk conceived the initial version of the presented idea. Both authors designed the experiment together. 
Utku T\"urk wrote the stimuli and conducted the experiment. The statistical analysis was carried out by Pavel Loga\v{c}ev; Utku T\"urk prepared all summaries and plots. Utku T\"urk wrote the first draft of the manuscript and both authors edited it subsequently. 

\section*{Data Availability Statement} %exclude

All data and R code used in the data analysis is available at \url{https://osf.io/hegmd/}.



\bibliographystyle{apacite}
\bibliography{lcn_short.bib}

\end{document}
 
