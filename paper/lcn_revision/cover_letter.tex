\documentclass[11pt]{letter}
\usepackage{blindtext}
\usepackage[T1]{fontenc}
\usepackage[utf8]{inputenc}

\name{Utku T\"{u}rk}
%\address{Bo\u{g}azi\c{c}i University}
%\signature{signature de l'expéditeur}
\date{\today} 

\begin{document}

\begin{letter}{Dear Prof. Lorraine K. Tyler,}

\opening{Together with Pavel Loga\v{c}ev (Bo\u{g}azi\c{c}i University), I would like to submit our work titled \emph{``Agreement Attraction in Turkish: The Case of Genitive Attractors''} to Language, Cognition and Neuroscience for consideration as a short article. }

In the manuscript, we test an alternative explanation of previously attested agreement attraction effects in Turkish (Lago et al., 2019, \textit{Linguistic Approaches to Bilingualism}). Lago and colleagues found that in Turkish genitive-possessive constructions similar to the English \textit{the singer's friends}, the genitive noun phrase (\textit{singer's}) can cause agreement attraction effects. This finding is surprising, as no such effects have been attested in comparable structures in English (Nicol et al., 2016, \textit{Front. in Psych.}).

Lago et al. argue that genitives can function as attractors in Turkish but not in English because genitive subjects exist in Turkish but not English. We test an alternative explanation, which is that their findings are due to a local ambiguity in their experiment, which constitutes a potential confound.

Having eliminated the potential confound, we were able replicate the agreement attraction effect in Turkish genitive-possessive structures. We thus demonstrate that agreement attraction effects occur in unambiguous genitive-possessive constructions in Turkish. 

This finding lends further credence to the idea that the observed differences between Turkish and English structures are indeed likely to be due to the presence of genitive subjects in the former but not in the latter as argued by Lago and colleagues.


\ps{
Sincerely,\\
Utku T\"{u}rk\\
Department of Linguistics\\
Bo\u{g}azi\c{c}i University\\
Turkey}
%\cc{nom des autres destinataires}
%\encl{pièces jointes}
\end{letter}
\end{document}